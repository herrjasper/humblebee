\section{Spellingbee -- Spelling correction}


\begin{frame}
 
\begin{figure}[h]
  \centering
  \scalebox{.7}{
    \begin{tabular}{lc@{\hspaceThis{XXXXXXXXXXX}}lrr}
      Correct form (frequency)& & Misspelled form & N & edit distance \\
      \hline
      &&\node{a}{\alert{u}bernimmt} & 81 & 1 \\
      &&\node{b}{überni\alert{n}mmt}& 6 & 1\\
      &&\node{c}{übernimm\alert{n}t}& 1 & 1\\
      &&\node{d}{überni\alert{e}mt} & 6 & 1\\
      &&\node{e}{\alert{ö}bernimmt} & 6 & 1\\
      &&\node{f}{übernim\alert{n}mt}& 5 & 1\\
      &&\node{h}{übernimm\alert{e}t}& 11 & 1\\
      \node{x}{übernimmt (297440)}&&\node{i}{überni\alert{eh}mt} & 2 & 2\\
      &&\node{j}{überni\alert{e}mt} & 6 & 1\\
      &&\node{k}{überni\alert{h}mt} & 17 & 1\\
      &&\node{l}{überni\alert{iiiiiiimmmmm}mmt} & 1 & 12\\
      &&\node{m}{überni\alert{o}mmt}& 4 & 1\\
      &&\node{n}{übernim\alert{rn}t}& 4 & 2\\
      &&\node{p}{überni\alert{u}mmt}& 2 & 1\\
      &&\node{q}{überni\alert{nn}t} & 2 & 2\\
      &&\node{r}{übernimm\alert{t}t}& 2 & 1\\
      &&\node{o}{\ldots}
    \end{tabular}
    \anodeconnect[r]{x}[l]{a}
    \anodeconnect[r]{x}[l]{b}
    \anodeconnect[r]{x}[l]{c}
    \anodeconnect[r]{x}[l]{d}
    \anodeconnect[r]{x}[l]{e}
    \anodeconnect[r]{x}[l]{f}
    \anodeconnect[r]{x}[l]{h}
    \anodeconnect[r]{x}[l]{i}
    \anodeconnect[r]{x}[l]{j}
    \anodeconnect[r]{x}[l]{k}
    \anodeconnect[r]{x}[l]{l}
    \anodeconnect[r]{x}[l]{m}
    \anodeconnect[r]{x}[l]{n}
    \anodeconnect[r]{x}[l]{p}
    \anodeconnect[r]{x}[l]{q}
    \anodeconnect[r]{x}[l]{r}
    \anodeconnect[r]{x}[l]{o}
  }
  Some spelling variants of German \textit{übernimmt} (`takes over') from DECOW2012.
\end{figure}

\end{frame}



\begin{frame}
  {Automatic spelling correction}

  \begin{enumerate}
  \item Identify potential misspellings
  \item Select candidate for replacement\label{select}
    \begin{enumerate}
    \item Produce candidates
    \item Rank candidates 
    \end{enumerate}
  \item Decision: replace ``misspelling'' with best candidate?\label{decide}
  \end{enumerate}


Order also reflects the complexity of these steps .

\end{frame}


\begin{frame}{Identifying potential misspellings}

  \begin{itemize}
  \item using \texttt{Enchant} library
  \item consult dictionaries: 
    \begin{itemize}
    \item \texttt{Aspell} en$_{GB}$
    \item \texttt{Aspell} en$_{US}$
    \item additional custom word list
    \end{itemize}
  \end{itemize}
  
\end{frame}


\begin{frame}{Ranking candidates}

\texttt{Enchant}/\texttt{Aspell} returns a ranked set of candidates.\\[1ex]

$\rightarrow$ re-rank candidates using context information\\[2ex] 


Language models: 
  \begin{itemize}
  \item from ``clean'' part of UKCOW2012 corpus (tokens)
  \item unigrams (4.2 m types)
  \item bigrams (74 m types)
  \end{itemize}  
\end{frame}



\begin{frame}{Candidate re-ranking}
  
\begin{tabular}{l@{~}l@{~}|l@{~}|l@{~}l}
   \textit{\ldots be necessarily brief} &  \textit{\alert<2>{but}} & efven & \textit{\alert<3>{that}} & \textit{being so does not excuse \ldots}\\
\hline
  &   & \textbf{even} &  & \\
  &   & eleven &  & \\
  &   & Evan &  & \\
  &   & oven &  & \\
  &   & \ldots &  & 
\end{tabular}

\vspace{1cm}

Conditional probabilites:\\[1ex]

\pause

P(\textbf{candidate}|\alert<2>{preceding})

P(\alert<3>{following}|\textbf{candidate})


\end{frame}


\begin{frame}{Ranking: Evaluation}
  

Best predictor: product of conditional probabilities\\


\begin{itemize}
\item Test set: 1007 potential mispellings, 85 real misspellings (8.4\%)
\item<2-> Baseline: \texttt{Aspell}'s best candidate: 82.4\% correct
\item<3-> Our re-ranking: 88.2\% correct % accuracy in candidate selection (trigram)
\end{itemize}

\visible<4->{


But: replacement is a rare event.

\begin{itemize}
\item real misspellings of capitalized words: 1.82\%
\item do not flag capitalized mid-sentence tokens as misspellings
\item real misspellings among remaing tokens: 32.17\%
\end{itemize}
}
\end{frame}



\begin{frame}{Replace or leave as is?}

\begin{itemize}
\item extract more information about candidate and misspelling
\item use balanced training set (50\% replace, 50\% leave-alone)
\item model the decision with a logistic regression
\end{itemize}

\end{frame}




\begin{frame}{Logistic regression: predictors}

Model trained on 310 instances.

  \begin{itemize}
  \item \alert<2->{\texttt{Aspell}'s original rank}
  \item \alert<2->{edit distance}
  \item number of alternative candidates
  \item \alert<2->{capitalization of misspelling}
  \item frequency of candidate in document
  \item \alert<2->{frequency of misspelling in document}
  \item bigrams: conditional probabilities misspelling
  \item bigrams: conditional probabilities candidate
  \item product of bigram conditional probabilities
  \end{itemize}

\end{frame}


\begin{frame}{Model evaluation}

On training data:\\[1ex]

  \begin{tabular}{lr}
   \% corr. & .81 \\
   precision & .75\\
   recall & .92 \\
   F1  & .83
  \end{tabular}

\pause
\vspace{1cm}


On unseen test data (unbalanced: 30\% replace)\\[1ex]

\begin{tabular}{lr}
\% corr. & .63\\
pecision & .45\\
recall   & .77\\
F1       & .56
\end{tabular}

\end{frame}



\begin{frame}{Summary spelling correction}

  \begin{itemize}
  \item context information helps chosing the right candidate
  \item toughest problem with web data: replace vs.\ leave-alone
  \item so far, the model does not generalize well to unseen data
  \end{itemize}

\pause

Next steps:

\begin{itemize}
\item restrict domain to more predictable cases
\item incorporate noisy channel model for frequent misspellings
\item use POS information
\end{itemize}


  
\end{frame}
